\begin{abstract}


The use of optimization in computational fluid dynamics for  engineering situations is a trending topic that is gaining a lot of attention. This study was focused on applying an optimization method to different computer fluid dynamics cases. Instead of using adjoint methods, genetic algorithms were used for the optimization process. The research was conducted with three different cases: vortex shedding in a cylinder wake (for minimizing the oscillations), diffuser inlet design (for maximizing the pressure ratio and Mach number) and airfoil shape optimization (different optimization configurations were tested). For these three cases, multiobjective optimization was imposed, having Pareto fronts with solutions that coexist in a trade-off situation rather than a single optimum point. The simulations produced promising results, showing that the approach is more than viable: it is a robust, adaptable and versatile way of optimizing engineering systems. Apart from the high level of parallelization and the automation of the procedure, the straightforward implementation allows that subtle changes in the code will optimize very different cases. Although the Pareto front was determined for the different cases, future work should be performed to increase the convergence of the method while reducing the number of simulations, given that they are a very time-consuming operation. 


\end{abstract}