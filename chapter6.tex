\chapter{Conclusions and further developments}

As it can be seen, genetic algorithms are a great approach to all kind of optimization problems: from CFD cases with multiobjective optimization to single objective optimization of other engineering systems. The results obtained show that this approach is (although time-consuming) efficient compared with other approaches. Genetic algorithms can be not very efficient, computationally speaking. But they are the only approach to some problems, as global optimization, because methods like gradient search locate only local minimum. Genetic algorithms may also be applied to black-box functions, where only an evaluation from the function is required. 

The results shown in Figures \ref{fig:genForDifusser}, \ref{fig:genForCLCD} and \ref{fig:genForLD} have the Pareto front correctly captured with the NSGA-II. However, there are zones in the Pareto front that are not correctly covered, leaving gaps in the non-dominated set and having areas with sparse individuals combined with very populated areas. There are different ways to improve these behaviors and before applying the algorithm to other cases, a refinement should be done to eliminate these discrepancies \cite{chichakly2013improving}, \cite{yuan2014improved}. However, these problems may be due to a little number of individuals and/or generations, having that larger numbers will give a more detailed and evenly distributed Pareto front.

The algorithm has proved to be robust in the sense that if there is a function space without a Pareto front, it will converge to the most optimal solutions, as done in the cylinder case. Nevertheless, this lack of Pareto front may also be due to a small population size. Cylinder case also requires a more in-depth analysis to extract more information from the data, such as the frequencies of the oscillations. With this detailed analysis, other parameters may be chosen for achieving better performance in the optimization and smaller oscillations in the flow.

Complex search spaces have been also considered within this analysis and the results show that this genetic algorithm may perform optimization in constrained problems. Although the constraints were only imposed in the search space, the NSGA-II also allows the use of constraints in the function space \cite{deb2002fast}.

The adaptability of the code is one of the most powerful characteristics of this optimization method. Slight changes in the code (just replacing some lines in the scripts) will allow the optimization of different objectives, as it was shown with the airfoil case. This makes the code versatile and open to a broad variety of possibilities when facing new optimization problems.

Although genetic algorithms have been presented as one of the better solutions, they also have problems and limitations. The most evident one is that GA are heuristic methods that only achieve an approximation of the actual solution. Due to the stochasticity of the algorithm, each run will return a slightly different approximation of that solution. However, finding the \say{optimal} solution may not be feasible and the solution obtained with genetic algorithms may consist of a robust solution for multi-objective purposes. 

There are other algorithms that may also be implemented for CFD optimization, such as evolutionary algorithms or particle swarm optimization. All these new machine learning techniques will be of high importance in the future of computer fluid simulation and the optimization of cases, codes, and shapes. 

Another major improvement that may be done in the application of genetic algorithms in the Computer Fluid Dynamics field is data management. Each simulation returns a huge amount of data (roughly $50\ Gb$ per case) and everything is reduced to just some bytes that contain the fitness value. Taking advantage of all the data of the simulations may be used for improving the convergence of the algorithm without being necessary the use of more generations or individuals. 

Another update in the code can be a higher parallelization level. Evolutionary algorithms have the advantage to be easily and efficiently parallelizable \cite{thevenin2008and}. The code only performs the CFD simulations in a parallel fashion. However, fitness evaluation (extracting data from the simulations) is also a very computer demanding operation. If this process was also run in parallel instead of in serial processors, the whole algorithm will run in shorter amounts of time. The big bottleneck of genetic algorithms in computer fluid dynamics is exactly this: the evaluation time of each individual is high when repeating it generation after generations. Analysis like processor convergence are more than justified to reduce as much as possible the simulation time.

The use of genetic algorithms in 3D simulations is possible, but the increase in computational time make it an unfeasible option for this report. The amount of data from a 3D simulation will also be way larger than the data of a 2D simulation, being necessary more space to handle all the files. Apart from computational limitations, the use of GA in 3D cases will also be a topic of high importance in the future.

Finally, the algorithm has been implemented for 2 variables and 2 objective functions. However, the real potential of the genetic algorithm lays on cases with more than two variables, where the search space is complex and can not be visualized in 2D. This search will also require higher computational resources and more time will be spent in doing simulations. In cases with two search space variable, the results may seem more or less intuitive and the Pareto front may be \say{located} and related from the search space to the function space. In the case of tenths of variables, this visualization will be a little trickier and the direct relationship between search and function spaces will not be that straightforward, being impossible to tweak the search space variables by hand without the use of a genetic algorithm. 